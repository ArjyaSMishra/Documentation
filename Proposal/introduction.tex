\section{Introduction}\label{sec:introduction}
In recent times \textit{bidirectional transformation} (denoted by "bx") has gained importance in the field of Model-Driven Software Development \cite{bx-grace} \cite{bx-dagstuhl}. BX is a technique used to synchronize two (or more) instance of different meta-models. Both models are related, but don't necessarily contain the same information. Changes in one model thus lead to changes in the other model \cite{bx-grace}. 
\newline\newline\textit{Bidirectional transformation} is used to deal with scenarios like:
\begin{itemize}
	\item {change propogation to user interface as a result of underlying data changes}	
	\item {synchronization of business/software models}
	\item {refreshable data-cache incase of database changes}
\end{itemize}
    and many more....
\newline\newline Bx community has been doing research and development work in many fields like software development, database, mathematics and many more to increase awareness and to reach more people \cite{bx-dagstuhl}\cite{bx-grace}. As a result, many kinds of bx tools are being developed, e.g., eMoflon \cite{emoflon-part4}, Echo \cite{echo}. These bx tools are based on various approaches, such as, graph transformations, bidirectionalization, update propagations \cite{bx-community} and can be used in different areas of application.
\newline\newline One basic problem that still persists is that the knowledge of and awareness for "bx" is limited. In this thesis, my goal is to design and implement an interactive demonstrator based on an example. An existing bx tool will be used as a part of the demonstrator to realize \textit{bidirectional transformation}. The final prototype will be interactive and easily accessible to users to help them understand the potential, power and limitations of bx.
\newline\newline This proposal is structured as follows: Section 2 describes the problem that I will address in my thesis. Section 3 presents the current state of the art and existing pain points. Based on this, in Section 4, research questions will be explained that I will be trying to solve in my thesis. Section 5 describes the research method. Section 6 describes the examples that can be implemented and the architecture design. Section 7 presents the time plan. Finally, Section 8 presents a tentative structure of the final thesis.


