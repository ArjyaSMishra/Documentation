\section{Requirements}\label{sec:requirements}
\subsection{Objectives}\label{subsec:objectives}
The goal of this thesis is to explore the fundamental and technical challenges involved in implementing a demonstrator for bx tools.
\newline\newline This thesis aims at answering these main research questions:
\\\textbf{\textit{RQ1}} -- What are the core requirements for implementing a successful bx demonstrator ?\\
\\\textbf{\textit{RQ2}} -- What kind of interactivity and to what extent is it required in the bx demonstrator ?\\
\\\textbf{\textit{RQ3}} -- Which goals can be particularly well addressed in a bx demonstrator and why ?\\
\\\textbf{\textit{RQ4}} -- To what extent is such a bx demonstrator reusable?\\
RQ4 can be split into the following sub-questions:
\\\textbf{\textit{RQ4.1}} -- Is the implementation of the demonstrator bx tool-specific ?\\
\\\textbf{\textit{RQ4.2}} -- Is the implementation of the demonstrator example-specific?\\
\\\textbf{\textit{RQ4.3}} -- What part(s) of the demonstrator can be reused in implementing a different example ?
\newline\newline All of my work is directly or indirectly related to the above research questions.

\subsection{Learning Goals}\label{subsec:learninggoals}
Along with exploring the challenges as described in Section \ref{subsec:objectives}, I am also focussing on teaching some basic concepts to the user about bx in the process of trying/playing with the demonstrator.

What am I trying to teach?
\begin{itemize} 
	\item {Bidirectional is not always bijective.} 
	\item {Not all changes can be propagated. In this case, consistency needs to be preserved.}
	\item {In BX, Challenge is to avoid or minimise information loss.}
	\item {Current limitation - you can only change one side.}
	\item {Synchronisation is interactive (to handle non-determinism).}	
\end{itemize}

\subsection{End Result}\label{subsec:endresult}
As my thesis is more focussed on the implementation of a demonstrator for consistency management based on a bx tool, following are the end results I am trying to achieve:
\begin{itemize} 
	\item {Reduce the installation time of the bx tool by making the demonstrator available online. Hence, the user is just a click away to try the bx tool and needs nothing to install on his/her machine.} 
	\item {Demonstartor should be interactive and fun to play with.}
	\item {User should be able to learn/understand the concepts of bx as described in Section \ref{subsec:learninggoals}.}
\end{itemize}

\subsection{Implementation Phases}\label{subsec:implementationphases}
My work is based on an \ac{evolutionary case study} which focuses on designing and implementing a successful bx tool demonstrator. My entire work cycle is described in the following paragraphs.
\paragraph{Case Study}

\paragraph{Examples and Implementation}

\paragraph{Evaluation}








