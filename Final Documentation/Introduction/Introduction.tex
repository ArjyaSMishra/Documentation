\section{Introduction}\label{sec:introduction}
 \textit{Bidirectional transformation} (denoted by "bx") is a technique used to synchronize two (or more) instance of different meta-models. Both models are related, but don't necessarily contain the same information. Changes in one model thus lead to changes in the other model \cite{bx-grace}. Bx makes sure that two models that can change over time have to be kept constantly consistent with each other.
\newline\newline\textit{Bidirectional transformation} is used to deal with scenarios like:
\begin{itemize}
	\item {change propagation to the user interface as a result of underlying data changes}	
	\item {synchronization of business/software models}
	\item {refreshable data-cache incase of database changes}
	\item {consistency management between two artifacts by avoiding data loss}
\end{itemize}
    and many more....

\subsection{Contribution}\label{subsec:contribution}    
Bx community has been doing research and development work in many fields like software development, database, mathematics and much more to increase awareness and to reach more people \cite{bx-dagstuhl}\cite{bx-grace}. As a result, many kinds of bx tools are being developed, e.g., eMoflon \cite{emoflon-part4}, Echo \cite{echo}. These bx tools are based on various approaches, such as graph transformations, bidirectionalization, update propagations \cite{bx-community} and can be used in different areas of application.

\subsection{Problem Statement}\label{subsec:probstmt}
\textit{Bidirectional transformation} is an emerging concept. In the past, many efforts have been made by conducting international workshops, seminars and through experiments conducted by developers / bx community to identify its potential. Also, in addition to the development of bx tools and bx language, benchmarks are being created for bx tools for systematic comparison \cite{benchmark-BX}.
\newline\newline  Although a significant amount of work has been done in this field, some basic problems still remain:
\begin{itemize}
	\item {Reachability to relevant communities is not significant due to the absence of a common vocabulary for bx across research disciplines \cite{bx-theoryandappl}. Seminars are still conducted for exchanging ideas in different communities to define a common vocabulary of terms and properties for bx \cite{bx-dagstuhl}.}	
	\item {Bx tools and their applicability is still not widely known even in the developers' communities. Due to the existing conceptual and practical challenges associated with configuring/trying a bx-tool for knowing its potential, using bx, bx-tools in building software systems, many developers and researchers are still using non-bx transformation tools to achieve properties which can be easily supported by bx-tools \cite{bx-theoryandappl}.}
	\item {Absence of a simple yet interactive bx tool demonstrator to depict the potential of \textit{bidirectional transformation} over preferred non-bx tool demonstrators among developers' and researchers' communities \cite{bx-theoryandappl}.}
\end{itemize}

\subsection{Solution Strategy}\label{subsec:solution}
To solve the problems as described in Section \ref{subsec:probstmt}, in this thesis, my goals are as follows:
\begin{itemize} 
\item {Design and implement an interactive demonstrator.} 
\item {Spreading the basic concepts of bx to a wide audience and making them accessible and understandable.}
\end{itemize}
An existing bx tool will be used as a part of the demonstrator to realize \textit{bidirectional transformation}. The final prototype will be interactive and easily accessible to users to help them understand the potential, power and limitations of bx.

\subsection{Structure}\label{subsec:structure}
This document is structured as follows: 

Chapter 1 (introduction) contains the introduction about the thesis topic and
motivation behind the making of this thesis with a solution strategy.

Chapter 2 describes the related work that has been done on bx in last few years and the related problems.

Chapter 3 explains the research questions that I aimed at solving throughout my thesis, research method and implementation idea.

Chapter 4 describes the architecture(application framework) design in brief with related diagrams.

Chapter 5 provides the layer-wise implementation details of the application along with an application overview.

Chapter 6 contains the feedback from user groups, evaluation results and learning goals(based on research questions).

Last chapter summarizes all the work which was done as part of this thesis and draws useful conclusions followed by future work.

\section{Foundation}\label{sec:foundation}
In this chapter, I am going to describe some commonly used terminologies with respect to bx so that the reader doesn't have any problem in understanding the further chapters.

As already mentioned, "bx is a technique to maintain consistency(synchronize) between two(or more) related models".

\paragraph{Model}
\paragraph{Delta}
\paragraph{State}
\paragraph{Model Space}
\paragraph{BX}
\paragraph{Consistency}




