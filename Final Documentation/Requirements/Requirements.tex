\section{Requirements}\label{sec:requirements}
This chapter explains all the requirements needed to implement a successful demonstrator. Section \ref{subsec:functionalreq} describes all the functional requirements. Afterwards, Section \ref{subsec:nonfunctionalreq} deals with all the non-functional requirements. Then, Section \ref{subsec:choicesthreats} describes some of the choices that I made during research and  implementation work and its associated threats. 

\subsection{Functional}\label{subsec:functionalreq}
\paragraph{Learning Goals}
Along with exploring the challenges as described in Section \ref{subsec:objectives}, I am also focussing on teaching some basic concepts(functional requirements, referred to as \textbf{FREQ} from now on) to the user about bx in the process of trying/playing with the demonstrator.

What am I trying to teach?
\begin{itemize} 
	\item {\textbf{\textit{FREQ1}} -- Bidirectional is not always bijective.} 
	\item {\textbf{\textit{FREQ2}} -- Not all changes can be propagated. In this case, consistency needs to be preserved.}
	\item {\textbf{\textit{FREQ3}} -- In BX, Challenge is to avoid or minimise information loss.}
	\item {\textbf{\textit{FREQ4}} -- Current limitation - you can only change one side.}
	\item {\textbf{\textit{FREQ5}} -- Synchronisation is interactive (to handle non-determinism).}	
\end{itemize}
\subsection{Non-Functional}\label{subsec:nonfunctionalreq}
As my thesis is more focussed on the implementation of a demonstrator for consistency management based on a bx tool, following are some of the non-functional end results (referred to as \textbf{NREQ} from now on) I am trying to achieve:
\begin{itemize} 
	\item {\textbf{\textit{NREQ1}} -- Reduce the installation time of the bx tool by making the demonstrator available online. Hence, the user is just a click away to try the bx tool and needs nothing to install on his/her machine.} 
	\item {\textbf{\textit{NREQ2}} -- Demonstartor should be interactive and fun to play with.}
	\item {\textbf{\textit{NREQ3}} -- Demonstrator's example should not be too technical and rather be simple to attract more users. Also, user should be able to relate to the bx concepts through the example.}
	\item {\textbf{\textit{NREQ4}} -- Demonstrator should be able to guide the user in the whole playing/learning process.}
\end{itemize}
\subsection{Technical}\label{subsec:technicalreq}

\subsection{Choices and Threats}\label{subsec:choicesthreats}
The implementation process and the final prototype was driven by many choices and threats. For example, 
\begin{itemize} 
	\item {selection of the bx tool and the most suitable example to implement has impact on deciding the usefullness of the demonstrator.}
	\item {seletion of the bx tool and the example to implement is more or less influenced by the ideas given by my supervisor.} 
	\item {my decisions on designing the application's framework for implementation are impacted by the existing availability and usability issues of the finalized bx-tool.}
	\item {During evaluation, I might not have got proper feedback from my friends \& colleagues due to my acquaintance and time constraints.}
\end{itemize}











